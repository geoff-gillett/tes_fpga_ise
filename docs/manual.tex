%% TES digitiser manual
\documentclass{article}
\usepackage{amsmath}
\usepackage{xcolor}
\usepackage{booktabs}
\usepackage[T1]{fontenc}
\usepackage{upquote}
\usepackage{graphicx}
\usepackage{cite}
\usepackage{svg}
\setsvg{inkscape = inkscape -z -D}
\setsvg{svgpath = images/}

\usepackage{relsize}
\usepackage{subfloat}
\newcommand{\code}[1]{\texttt{#1}}
\usepackage{bytefield}
%\usepackage{register}
\usepackage{bitpattern}
\usepackage{mathdots}

\newsavebox{\fieldname}
\newlength{\fieldwidth}
\newsavebox{\bits}
\newsavebox{\bitlabel}
\newlength{\bitlabelwidth}
\settowidth{\bitlabelwidth}{\code{ field}}
\savebox{\bitlabel}{\raisebox{0.9\baselineskip}
{\makebox[0pt][r]{\makebox[\bitlabelwidth][r]{bit}}}}
\savebox{\bitlabel}
{\makebox[\bitlabelwidth][r]{field}\usebox{\bitlabel}\hspace{-12pt}}

\newcommand{\bitfield}[3]{
\savebox{\fieldname}{\code{#1}}
\settowidth{\fieldwidth}{\usebox{\fieldname}}
\savebox{\bits}{\raisebox{.9\baselineskip}
{\makebox[0pt][l]{\makebox[\fieldwidth]{\code{#2}\hfill\code{#3}}}}}
\mbox{%
\colorbox[gray]{0.9}
{\mbox{\usebox{\bits}\usebox{\fieldname}}\rule[-.2\baselineskip]
{0pt}{1.7\baselineskip}}\rule[-.6\baselineskip]{0pt}{2.5\baselineskip}
\hspace{-18pt}}}

\newcommand{\bitflag}[2]{
\savebox{\fieldname}{\code{#1}}
\settowidth{\fieldwidth}{\usebox{\fieldname}}
\savebox{\bits}{\raisebox{0.9\baselineskip}
{\makebox[0pt][l]{\makebox[\fieldwidth][c]{\code{#2}}}}}
\mbox{%
\colorbox[gray]{0.9}
{\mbox{\usebox{\bits}\usebox{\fieldname}}\rule[-.2\baselineskip]
{0pt}{1.7\baselineskip}}\rule[-.6\baselineskip]{0pt}{2.5\baselineskip}
\hspace{-18pt}}} 

\newenvironment{fielddesc}
{\texttt\bgroup}
{\egroup}

\newenvironment{regfields}
{\usebox{\bitlabel}}
{}


\newenvironment{register}[2]
{\vspace{0.6\baselineskip}\begin{minipage}{\textwidth}%
\code{#1}

\code{address:}\code{#2}

\vspace{0.5\baselineskip}
\begin{regfields}}
{\end{regfields}\end{minipage}\vspace{0.5\baselineskip}}

\newcommand{\return}{\code{\textbackslash r}}
%\newconnand{\subs}[2]{\ensuremath{\mathrm{#1}_\mathrm{#2}}}


\setlength{\parindent}{0pt}
\setlength{\parskip}{.5\baselineskip}
\setlength{\fboxsep}{2pt}

\begin{document}
\bibliographystyle{unsrt}


\section{Superconducting Transition Edge Sensors}
A superconducting transition edge sensor (TES) is a micro calorimeter, a device
that measures particle energy via a change in temperature. All calorimeters 
consist of three 
sub-systems; an absorber, a thermometer for measuring the absorbers 
temperature change and cold bath weakly coupled to the absorber to cool and 
reset the calorimeter after particle absorption. 

The TES provided by our collaborators the National Institute of Standards and
Technology (NIST) in Boulder Colorado \cite{Lita:08} is engineered to detect
photons and all three calorimeter roles are fulfilled by a thin film of Tungsten
held in the phase transition between the normal and superconducting states.
Electrons act as both absorber and thermometer and phonons weakly couple the
electrons to the atomic lattice which provides the cold bath.

As a thin film tungsten exhibits type II superconducting properties, typified
by two critical currents.

% \begin{figure}[!hbt]
%   \includesvg[width=0.6\textwidth]{typeIIsuperconductor}
% \end{figure}
% \begin{figure}[!hbt]
%   \includesvg[width=0.6\textwidth]{biastrajectory}
% \end{figure}
\begin{figure}[!hpbt]
  \centering
  \includesvg[width=0.7\textwidth]{typeIIsuperconductor}
  \caption{
    Below the lower critical line $\mathrm{I}_{\mathrm{c}_{1}}$ the system is
    superconducting with zero resistance and the Meissner effect prevents
    magnetic flux entering the superconductor. In the transition region, between
    the two critical lines, flux quanta begin to penetrate and around each
    penetration a supercurrent loop forms. These loops, know as ``vortexes'', 
    generate a ``screening'' magnetic field opposing and cancelling the field 
    outside the loop. 
    The transition region is a mixture of both the normal and superconducting
    phases, islands of normal phase penetrated by quantised flux encircled and
    screened by a supercurrent loop surrounded by superconducting phase. In
    this region resistance is non-zero and dependent on vortex density??, a
    function of both temperature and current. Above $\mathrm{I}_{\mathrm{c}_2}$
    the system is fully normal and it resistance has its usual weak
    dependence on temperature only. (TODO: check terminology and facts).
    \label{fig:type2sc}
  }
\end{figure}

\newpage

To operate as a photon detector, the TES is cooled to approximately
$100\,\mathrm{mK}$ well bellow its critical temperature 
($\mathrm{T}_\mathrm{c}\sim\,150\,\mathrm{mK}$) and electrically biased into the
transition region between the two critical currents.
In this region the TES supports a mixture of both superconducting and normal
phases and has a non-zero resistance which is strongly?? dependant on both
electron temperature and current.

\subsection{Biasing and Readout}

\begin{figure}[!hpbt]
  \subfloat[Bias and readout cicuit]{%
     \includesvg[pretex=\relscale{0.5},width=0.5\textwidth]{circuit}
     \label{fig:circuit}
  }
  \subfloat[TES biasing]{
    \includesvg[width=0.5\textwidth]{sensorbias}
    \label{fig:sensorbias}
  }
  \caption{
    (a) The TES placed in parallel with a small shunt resistance forming a
    current divider biased by $\mathrm{V}_\mathrm{b}$. The sensor
    current ($\mathrm{I}_\mathrm{s}$) is inductively coupled to an array of DC
    SQUIDs which transduce $\mathrm{I}_\mathrm{s}$ to a voltage to be further
    amplified at room temperature. Voltage $\mathrm{V}_\mathrm{f}$ sets the
    operating point on the SQUIDs voltage-flux response to maximise transduction
    gain.
    (b) As $\mathrm{V}_\mathrm{b}$ increases $\mathrm{I}_\mathrm{s}$
    increases without heating the sensor until it
    exceeds $\mathrm{I}_{\mathrm{c}_1}$ where the electron temperature 
    begins to rise above the lattice temperature $\mathrm{T}_\mathrm{bath}$
    due to the non-zero resistance. $\mathrm{V}_\mathrm{b}$ is adjusted to bring
    the TES to an operating point within the transition region with some
    quiescent current $\mathrm{I}_\mathrm{q}$. $\mathrm{I}_\mathrm{q}$ is stable
    under small changes in temperature due to electrothermal feedback (see
    text).
  }
\end{figure}

The circuit of figure \ref{fig:circuit} is used to voltage bias the TES into the
transition region. 
Voltage biasing is preferred over a constant sensor current
$\mathrm{I}_\mathrm{s}$ as this facilitates electro-thermal feedback (ETF)
reducing the time taken for the sensor to return to a quiescent state after
photon absorption. 
ETF occurs due to negative feedback between temperature and current, any rise
in the electron temperature increases TES resistance
($\mathrm{R}_\mathrm{TES}$) reducing current in the sensor arm of the current
divider formed by $\mathrm{R}_\mathrm{TES}$, $\mathrm{R}_\mathrm{shunt}$ and
$\mathrm{R}_\mathrm{L}$ and this consequently reduces ohmic heating of the
sensor.
This reduction in heating cools the electrons faster than the weak phonon
coupling between the electrons and the lattice alone.
ETF stabilises the TES at operating and during a detection event keeps
power dissipation to the bath approximately constant \cite{SaeWooNam:1999hz}.

Absorption of a photon by the electrons rapidly heats them leading to a rapid
reduction in sensor current, this negative going current pulse is the basis of
the detection signal encoding the time and amount of energy absorbed (figure
\ref{fig:detection}).
Pulse depth and area increase in proportion to the energy of the detection event
until there is enough energy to drive the TES into the normal conduction phase.
At this point pulse depth saturates but energy information continues to be
carried in the area and length of the pulse with
reduced resolution\cite{2014JOSAB..31B..20L}.
(TODO: summarise the results of that paper.  Find and add quantitative values
for the intrinsic energy resolution, time resolution jitter etc for latter
comparison, with values measured with FPGA to understand how well the processing
extracts the information)


\begin{figure}[!hpbt]
\centering 
  \includesvg[width=0.8\textwidth]{detection}
  \caption{
    Absorption of photons by the electrons in the TES raises their temperature,
    quickly increasing resistance and reducing sensor current which returns to
    its quiescent level as the electrons re-cool.
    The depth and area of this negative going pulse encodes the detection
    energy and in the case of monochromatic light the number of photons
    detected.
    If enough energy is absorbed the TES will transition to the normal
    conducting state and saturate the pulse depth. The left panel illustrates
    the sensors trajectory in the superconducting phase diagram for different
    event energies, the right sensor current vs time.
  }
  \label{fig:detection}
\end{figure}


% \begin{figure}[!hpbt]
%   \centering
%   \includegraphics[width=0.8\textwidth]{images/input.png}
%   \caption{
%     Detection signal inverted and amplified at room temperature. Inset
%     typical detection pulse }
%   \label{fig:tesoutput}
% \end{figure}
% 
% \begin{figure}[!hpbt]
%   \centering
%   \includegraphics[width=0.8\textwidth]{images/baselinecorrected.png}
%   \caption{
%     baseline correction.
%   }
% \end{figure}
% 
% \begin{figure}[!hpbt]
%   \centering
%   \includegraphics[width=0.8\textwidth]{images/dsp.png}
%   \caption{
%     Digital filtering.
%   }
% \end{figure}
% 
\clearpage

\section{Processing the TES output signal}
After room temperature amplification, the signal is digitised by a DC coupled 14
bit analogue to digital converter (ADC) with a $250$ Mhz sample rate. ADC output
is processed by a Field programmable gate array (FPGA) which perform time and
energy measurements on the detection pulses. The FPGA expects positive going
pulses, and this can be  achieved by using an inverting amplifier or adjusting
$\mathrm{V}_\mathrm{f}$ (figure \ref{fig:circuit}) to give negative transduction
gain. DC coupling allows adjustment of the DC offset of the signal giving full
access $14$ bit dynamic range of the ADC.

(TODO: compare other methods and FPGA approach, add a brief introduction to
FPGAs).

Prior to measurement a signal conditioning pipeline removes signal noise
(figure \ref{fig:pipeline}), by first performing baseline correction then
applying two stages of filtering using digital signal processing (DSP) 
techniques.

\begin{figure}[!hpbt]
  \centering
  \includegraphics[width=\textwidth]{images/pipeline.png}
  \caption{
    Preparation for measurement - pre-processing pipeline; Top:TES output
    inverted and amplified.
    Middle:Baseline correction removes low frequency noise. 
    Bottom:Two stage digital filter,
    stage 1 smooths the the signal removing high frequency noise, stage 2
    calculates a smooth derivative.
  }
  \label{fig:pipeline} 
\end{figure}

\subsection{Baseline estimation and correction}
Low frequency baseline wander is corrected by monitoring the distribution of the
ADC output with a multi-channel analyser (MCA). 


A MCA is a circuit that captures the distribution of a signal by
counting signal values in a memory over some integration period, each
memory location representing a histogram bin. 
The MCA I have designed
also dynamically outputs the highest count and corresponding bin (most frequent
bin), ie the peak of the the distribution, this forms the basis of the baseline
estimation (figure \ref{fig:baselineestimation}). 

The estimation process is controlled by the following registers:
\begin{itemize}
  \item \code{baseline.offset}
  \item \code{baseline.threshold}
  \item \code{baseline.period}
  \item \code{baseline.count\_threshold}
  \item \code{baseline.average\_order}
\end{itemize} 

To conserve FPGA resources the baseline MCA is constructed with $2^{11}$ bins
covering one eighth of the dynamic range of the ADC, for correct operation the
quiescent signal level must be centered in the range $-2^{10}$ to $2^{10}+1$. 
The \code{baseline.offset} register is used to achieve this and its value is
subtracted from each unsigned $14$ bit ADC sample producing a signed $15$ bit
value. The register \code{baseline.threshold} sets an upper limit on values that
will be recorded in the histogram and can exclude detection peaks from the
baseline estimation.
In order to capture the dynamics of the distribution the MCA counters must be
periodically reset to zero, this is controlled by register
\code{baseline.period}. During this period the 
highest count and corresponding bin currently in the MCA are available. This
most frequent bin is passed to an averaging circuit when it is \emph{new}, ie
when either the bin or the count value change, and when the count is above 
the value held in the \code{baseline.count\_threshold} register. The baseline
estimate is a ($15.1$ bit) average over $2^{n}$ bin values , where $n$ is the
value of the \code{baseline.average\_order} register. (TODO: Add a flowchart to
describe this process clearly?)

(TODO:Compare simple averaging over the integration time with the MCA
technique, preliminary investigations showed that the uniploar nature of the
signal caused the average to drift up as rate increased and the MCA performed
better. I have never had enough data to a quantitative
analysis. It maybe possible to remove the
baseline MCA, save resources and simplify the circuit).

(TODO: Establish where the baseline noise is coming from, my current hypothesis
is that it must be LF noise on $\mathrm{V}_\mathrm{f}$ which should be more
sensitive than $\mathrm{V}_\mathrm{b}$. The relative sensitivity could be
calculated from a decent quantitative measurement of the SQUIDS
$\mathrm{V}-\phi$ response. The only other interaction I can think of is
stray flux at the squid, should check for cross talk at the SQUIDS.)

\begin{figure}[!hpbt]
  \centering
  \includegraphics[width=\textwidth]{images/baselineestimation.png}
  \caption{
    The baseline estimate is the most frequent value of the digitised signal.
    The left panel shows the digitised signal with three MCA integration periods
    highlighted, the right shows the three distributions corresponding to the
    integration periods.
  }
  \label{fig:baselineestimation}
\end{figure}

\clearpage

\subsection{Digital filtering}

A linear time-invariant (LTI) system is completely represented by its impulse
response, the system output is the convolution of the input and
impulse response, so using a circuit that can convolve two signals it possible
to emulate any LTI system. This is particularly useful in signal processing
where the impulse response, called the filter kernel, is designed to emulate a
filter with some desired frequency charateristics. 

In digitised (discrete-time) systems convolution becomes a sum
\begin{equation*}
  y[n]=\sum_{k=0}^{N-1}h_k x[n-k]
\end{equation*}
 
where $x[n]$ is the input sequence, $y[n]$ the output and $c_k$ the kernel
coefficients, for finite kernels convolution is reasonably straight
forward to implement in hardware.

\begin{figure}[!hpbt]
  \centering 
  \includesvg[width=0.8\textwidth]{systolicFIR}
  \caption{
    Schematic representation of a direct form finite impulse response (FIR)
    filter using pipelined systolic multiply-accumulate architecture, the FPGA
    contains dedicated DSP slices for easy implementation.
    At the top is an $N$ tap delay line ($z^{-n}$ is the z-transform
    representation of a delay by $n$). The output of each tap is multiplied by
    its corresponding kernel coefficient and these partial products are added
    together producing the output sample. 
  }
  \label{fig:FIR}
\end{figure}

The signal conditioning pipeline includes two $23$ tap FIR filters the first
stage smooths the signal the second calculates the derivative. The kernels of
both filters are re-loadable and can be updated without changes to the
hardware, the default magnitude responses are shown in figure
\ref{fig:filterresps}. (TODO: add description of reload procedure.)

\begin{figure}[!hpbt]
  \centering
  \includesvg[width=\textwidth]{filter_resps}
  \caption{
    Magnitude responses of the default FIR filters, left smoother stage, right
    differentiator stage.}
  \label{fig:filterresps}
\end{figure}

\clearpage

\section{Measurement}

Detection pulses are converted into fixed length event structures for
transmission to the host computer. Three event types peak pulse and area plus
a tick event. Each event structure contains a $16$ bit relative
time-stamp recording time elapsed since the last event, this time-stamp
saturates after $131068$ ns. The tick event contains a $64$ bit time-stamp that
effectively never saturates (2338 years), the time between tick events is set
with the register \code{tick\_period} at setting of $65355$ enables a completely
time resolved detection record.

\bpLittleEndian
\bpNumberBitsAbove
%\bpNumberFieldsTwice
%\bpSetTickHeight 0
eFlags:
\bitpattern[startBit=15]{PC}[4]R[1]C[3]{TT}[2]{HT}[2]{DT}[2]T[1]N[1]/

\begin{itemize}
  \item PC: peak count or peak number
  \item R: height/time measurements are relative to the start point.
  \item C: channel
  \item TT: timing type
  \item HT: height type
  \item ET: detection type
  \item T: is tick
  \item N: new window
\end{itemize}


Detection Types:\\
Peak event: \\
\bitpattern[startBit=63]{height}[16][8]{rise time}[16][8]{eFlags}[16][8]
{rTime}[16][8]/

Area event: \\
\bitpattern[startBit=63]{area}[32][16]{eFlags}[16][8]{rTime}[16][8]/

Pulse event: \\
\bitpattern[startBit=63]{size}[16][8]{plength}[16][8]{eFlags}[16][8]
{rTime}[16][8]/
\\
\bitpattern[startBit=63,noBitNumbers]{area}[32][16]{pthresh}[16][8]
{sThresh}[16][8]/
\\
\bitpattern[startBit=63,noBitNumbers]{height}[16][8]{minima}[16][8]
{rise time}[16][8]{pTime}[16][8]/
peak
\\
Number of slots set by \code{max\_peaks} register. $\vdots$
\\
\bitpattern[startBit=63,noBitNumbers]{height}[16][8]{minima}[16][8]
{rise time}[16][8]{pTime}[16][8]/

\clearpage

Tick event:\\
\bitpattern[startBit=63]{period}[32][16]{tFlags}[16][8]{rTime}[16][8]/
\\
\bitpattern[startBit=63,noBitNumbers]{full time-stamp}[64][32]/
\\
\bitpattern[startBit=63,noBitNumbers]{overflow and error flags}[64][32]/
  
\subsection{Peak extraction}
Two registers \code{slope\_threshold} and \code{pulse\_threshold} are used to
control the measurement process and zero crossings of the slope establish the
local extrema in the conditioned detection signal.
When the slope makes a positive going crossing of \code{slope\_threshold} the
signal between the previous minimum and the next maximum is
considered a peak. Three different heights can be recorded; the raw height $h$
the difference between the maximum and minimum, a constant fraction of the raw
height $(1-f)h$ where $f$ is the value in the \code{constant\_fraction}
register and the integral of the slope between the zero crossings for the
minimum and minimum. A time-stamp can be generated at $5$ points; the starting
minimum, when the signal rises $fh$ above the minimum, when the slope crosses
\code{slope\_threshold}, when the signal crosses \code{pulse\_threshold} and at
the peak maximum.

\begin{figure}[!hpbt]
  \centering
  \includesvg[width=0.8\textwidth]{peakextraction}
  \caption{
    Peak extraction
  }
  \label{fig:peakextraction}
\end{figure}


\begin{figure}[!hpbt]
  \centering
  \includesvg[width=0.8\textwidth]{pulsemeasurement}
  \caption{
    pulse measurement
  }
  \label{fig:pulsemeasurement}
\end{figure}

\begin{figure}[!hpbt]
  \centering 
  \includesvg[pretex=\relscale{0.3},width=0.8\textwidth]{design}
\end{figure}

\clearpage

\subsection{Multi channel analyser}
The multichannel analyser capture the distribution of the measurements
\subsection{measurement}
Threshold and zero-crossings are used to generate measurements of the detection
pulses. (needs figures)
\subsection{Baseline estimation}
A cut down MCA is used to track the distribution of the raw signal. The peak of
this distribution is considered the baseline. 

\subsection{Stream output}
Measurements and distributions are streamed to the host PC as Ethernet frames
over a point to point connection. 

% \begin{figure}
%   \includesvg[width=0.7\textwidth]{channel}
% \end{figure}
% 
% \begin{figure}
%   \includesvg[width=0.7\textwidth]{raw}
% \end{figure}
% 
% \begin{figure}
%   \includesvg[width=0.7\textwidth]{baselinecorrected}
% \end{figure}

\section{Registers}


\subsection{Low level serial protocol}

The register IO protocol is compatible with the AMBA AXI Lite specification and
is currently transported over a RS232 serial connection via the Silicon Labs
CP210x USB to UART Bridge on the development board.
Commands and responses are encoded as \emph{ASCII hex characters}
\code{(0-9,A-F)} and terminated with a carriage return \return{} \code{(0x0D)}.

\subsubsection{Commands}

Commands are 19 characters in length including the terminator in this format

\begin{fielddesc}
"VVVVVVVVAAAAAAAA0C\return" 
\end{fielddesc}

where the 8 hex characters "\code{VVVVVVVV}" represent the 32 bit value to be
written, "\code{AAAAAAAA}" the 32 bit register address and "\code{C}" one of the
following command op-codes.

\begin{fielddesc}
\begin{tabular}{l l}
"1" & write register \\
"2" & read register \\
"3" & reset (warm reboot of FPGA)
\end{tabular}
\end{fielddesc} 


The value part is ignored in a read command but must be present, and must be
"\code{00000000}" for a valid reset.

\subsubsection{Responses}

After a command is processed a 3 or 11 character response is returned

\begin{fielddesc}
\begin{tabular}{l l}
"RC\return"                  & write response \\
"VVVVVVVVRC\return"          & read or reset response \\
\end{tabular}
\end{fielddesc}

where "\code{C}" is the op-code being responded to, "\code{R}" the response
code and "\code{VVVVVVV}" the returned value.
The 4 bit response code indicates errors during the command

\begin{regfields}
\bitflag{non-hex}{3}
\bitflag{bad length}{2}
\bitfield{AXI response}{1}{0}
\end{regfields}

the \code{non-hex} bit indicates that illegal characters were found in
the command, the \code{bad length} bit indicates the received command was not 19
characters long, and the 2 bit \code{AXI response} flags any processing errors 

\begin{fielddesc}
\begin{tabular}{l l}
00 & OKAY \\ 
10 & SLVERR \\
11 & DECERR \\
\end{tabular}
\end{fielddesc}

A \code{DECERR} indicates an unknown address, and a \code{SLVERR} a command
error, eg writing to a read-only register. The value in a reset response is the
FPGA features register.

\subsubsection{Addresses}

\begin{regfields}
\bitfield{\makebox[0.125\textwidth][c]{route}}{31}{24}
\bitfield{\makebox[0.5\textwidth][c]{24 bit address}}{23}{0}
\end{regfields}

The MSB of the address is used to route the command to various sub-systems
within the design

\begin{fielddesc}
\begin{tabular}{l l}
"0C" & registers for channel "C" \\ 
"10" & global registers \\
"20" & SPI communication \\
\end{tabular}
\end{fielddesc}

the SPI channels communicate with the chips on FMC daughter board,
the four ADS62P49 ADC chips and the Analogue Devices AD9510 clock distribution
chip. SPI address are of the form

\begin{regfields}
\bitfield{\makebox[30pt]{20}}{31}{24}
\bitfield{reserved}{23}{16}
\bitfield{chip select}{15}{8}
\bitfield{address}{7}{0}
\end{regfields}

with a FMC108 installed the chip selects are

\begin{fielddesc}
\begin{tabular}{l l}
bit 0 & ADS62P49 chip 0 (channels 0 and 1)\\
bit 1 & ADS62P49 chip 1 (channels 2 and 3)\\
bit 2 & ADS62P49 chip 2 (channels 4 and 5)\\
bit 3 & ADS62P49 chip 3 (channels 6 and 7)\\
bit 4 & AD9510 chip \\
\end{tabular}
\end{fielddesc}

\clearpage
\subsection{Global registers}

\begin{register}{cpu\_version}{0x01000000 READ ONLY}
\bitfield{ year }{31}{28}
\bitfield{month}{27}{24}
\bitfield{ day }{23}{16}
\bitfield{hour}{15}{8}
\bitfield{minute}{7}{0}
\end{register}
The build date for the software running on the main picoblaze CPU that controls
register IO, \code{year} is modulo 16.

\begin{register}{HDL version}{01000001}
	\bitfield{reserved}{31}{24}
	\bitfield{git short SHA-1}{23}{0}
\end{register}
The short SHA-1 for the git commit of the HDL source.

\begin{register}{FPGA features}{01800000 READ ONLY}
\bitflag{AD9510 status}{10}
\bitflag{FMC power}{9}
\bitflag{ FMC }{8}
\bitfield{ ADC }{7}{4}
\bitfield{Channels}{3}{0}
\end{register}

\begin{fielddesc}
  \begin{tabular}{l l}
    AD9510 status & See data sheet \\
    FMC power & FMC card booted with out error and is powered up\\
    FMC present & FMC card is installed \\ 
    ADC & number of adc chips on the FMC card\\
    DSP & number of processing channels instantiated in the FPGA.\\
  \end{tabular}
\end{fielddesc}

\begin{register}{ADC enable}{01000080}
\bitfield{reserved}{31}{8}
\bitfield{enable bits}{7}{0}
\end{register}
Each enable bit corresponds to the same ADC channel number, disabling an ADC
channel puts it in stand-by mode.

\begin{register}{Channel enable}{01000100}
\bitfield{reserved}{31}{8}
\bitfield{enable bits}{7}{0}
\end{register}
Each enable bit corresponds to the same processing channel, disabling a channel
prevents any information from that channel appearing in the event stream.

\begin{register}{Tick period}{01000010}
\bitfield{tick period}{31}{0}
\end{register}
The length of time between tick events in system clocks ($4 ns$).

\begin{register}{Tick latency}{01000020}
\bitfield{tick latency}{31}{0}
\end{register}
The maximum allowed to elapse since the last tick event was put in the event
stream before the event buffer is flushed.
 
\section{VHDL Notes}

\begin{verbatim}
Reset sequence
CPU reset - main CPU
reset0    - AD9510, ADCs, ADC FIFOs, 
            Global and channel registers, 
            Channel CPUs, ethernet_framer
reset1    - eventstream_mux, MCA
reset2    - measurement_unit
\end{verbatim}

\bibliography{thesis}
\end{document}



