%% TES digitiser manual
\documentclass{article}
\usepackage{amsmath}
\usepackage{xcolor}
\usepackage{booktabs}
\usepackage[T1]{fontenc}
\usepackage{upquote}
\usepackage{svg}
\setsvg{inkscape = inkscape -z -D}
\setsvg{svgpath = images/}

\newcommand{\code}[1]{\texttt{#1}}

\newsavebox{\fieldname}
\newlength{\fieldwidth}
\newsavebox{\bits}
\newsavebox{\bitlabel}
\newlength{\bitlabelwidth}
\settowidth{\bitlabelwidth}{\code{ field}}
\savebox{\bitlabel}{\raisebox{0.9\baselineskip}
{\makebox[0pt][r]{\makebox[\bitlabelwidth][r]{bit}}}}
\savebox{\bitlabel}
{\makebox[\bitlabelwidth][r]{field}\usebox{\bitlabel}\hspace{-12pt}}

\newcommand{\bitfield}[3]{
\savebox{\fieldname}{\code{#1}}
\settowidth{\fieldwidth}{\usebox{\fieldname}}
\savebox{\bits}{\raisebox{.9\baselineskip}
{\makebox[0pt][l]{\makebox[\fieldwidth]{\code{#2}\hfill\code{#3}}}}}
\mbox{%
\colorbox[gray]{0.9}
{\mbox{\usebox{\bits}\usebox{\fieldname}}\rule[-.2\baselineskip]
{0pt}{1.7\baselineskip}}\rule[-.6\baselineskip]{0pt}{2.5\baselineskip}
\hspace{-18pt}}}

\newcommand{\bitflag}[2]{
\savebox{\fieldname}{\code{#1}}
\settowidth{\fieldwidth}{\usebox{\fieldname}}
\savebox{\bits}{\raisebox{0.9\baselineskip}
{\makebox[0pt][l]{\makebox[\fieldwidth][c]{\code{#2}}}}}
\mbox{%
\colorbox[gray]{0.9}
{\mbox{\usebox{\bits}\usebox{\fieldname}}\rule[-.2\baselineskip]
{0pt}{1.7\baselineskip}}\rule[-.6\baselineskip]{0pt}{2.5\baselineskip}
\hspace{-18pt}}}

\newenvironment{fielddesc}
{\texttt\bgroup}
{\egroup}

\newenvironment{regfields}
{\usebox{\bitlabel}}
{}

\newenvironment{register}[2]
{\vspace{0.6\baselineskip}\begin{minipage}{\textwidth}%
\code{#1}

\code{address:}\code{#2}

\vspace{0.5\baselineskip}
\begin{regfields}}
{\end{regfields}\end{minipage}\vspace{0.5\baselineskip}}


\newcommand{\return}{\code{\textbackslash r}}

\setlength{\parindent}{0pt}
\setlength{\parskip}{.5\baselineskip}
\setlength{\fboxsep}{2pt}

\begin{document}

\section{Transition edge sensors}
A Transition edge sensor (TES) is a micro calorimeter, a device that measures
particle energy via a change in temperature. All calorimeters consist of three 
sub-systems; an absortber, a thermometer for measuring the absorbers 
temperature change and a weak coupling of the absorber to a cold
bath to cool and reset the sensor after particle absorbtion. 

In the NIST (ref) near IR TES, all three roles are fullfilled by a
thin film of Tungsten held in the quantum phase trasition between the normal and
superconducting states. The electron subsystem acts as
both absorber and thermometer and phonons weakly couple the electrons to the
tungsten atomic latice which provides the cold bath. 

As a thin film, tungsten becomes a type II superconductor with its phase diagram
displaying two critical magnetic fields (or equivilently currents).  

% \begin{figure}[!hbt]
%   \includesvg[width=0.6\textwidth]{typeIIsuperconductor}
% \end{figure}
% \begin{figure}[!hbt]
%   \includesvg[width=0.6\textwidth]{biastrajectory}
% \end{figure}
\begin{figure}[!hpbt]
  \centering
  \begin{minipage}[b]{0.48\textwidth}
    \includesvg[width=\textwidth]{typeIIsuperconductor}
    \caption{Phase Diagram}
  \end{minipage}
  \hfill
  \begin{minipage}[b]{0.48\textwidth}
    \includesvg[width=\textwidth]{biastrajectory}
    \caption{Biasing}
  \end{minipage}
\end{figure}

The TES package is held below its critical temperature and biased into the
transition region by passing a current through it. calc

\section{Biasing and Readout}

\section{Overview}
Main components 
DSP 
Measurement
MCA
Baseline estimation
blah

% \begin{figure}
%   \includesvg[width=0.7\textwidth]{channel}
% \end{figure}
% 
% \begin{figure}
%   \includesvg[width=0.7\textwidth]{raw}
% \end{figure}
% 
% \begin{figure}
%   \includesvg[width=0.7\textwidth]{baselinecorrected}
% \end{figure}

\section{Registers}

\subsection{Serial protocol}

The register IO protocol is compatible with the AMBA AXI Lite specification and
is currently transported over a RS232 serial connection via the Silicon Labs
CP210x USB to UART Bridge on the development board.
Commands and responses are encoded as \emph{ASCII hex characters}
\code{(0-9,A-F)} and terminated with a carriage return \return{} \code{(0x0D)}.

\subsubsection{Commands}

Commands are 19 characters in length including the terminator in this format

\begin{fielddesc}
"VVVVVVVVAAAAAAAA0C\return" 
\end{fielddesc}

where the 8 hex characters "\code{VVVVVVVV}" represent the 32 bit value to be
written, "\code{AAAAAAAA}" the 32 bit register address and "\code{C}" one of the
following command op-codes.

\begin{fielddesc}
\begin{tabular}{l l}
"1" & write register \\
"2" & read register \\
"3" & reset (warm reboot of FPGA)
\end{tabular}
\end{fielddesc} 

The value part is ignored in a read command but must be present, and must be
"\code{00000000}" for a valid reset.

\subsubsection{Responses}

After a command is processed a 3 or 11 character response is returned

\begin{fielddesc}
\begin{tabular}{l l}
"RC\return"                  & write response \\
"VVVVVVVVRC\return"          & read or reset response \\
\end{tabular}
\end{fielddesc}

where "\code{C}" is the op-code being responded to, "\code{R}" the response
code and "\code{VVVVVVV}" the returned value.
The 4 bit response code indicates errors during the command

\begin{regfields}
\bitflag{non-hex}{3}
\bitflag{bad length}{2}
\bitfield{AXI response}{1}{0}
\end{regfields}

the \code{non-hex} bit indicates that illegal characters were found in
the command, the \code{bad length} bit indicates the received command was not 19
characters long, and the 2 bit \code{AXI response} flags any processing errors 

\begin{fielddesc}
\begin{tabular}{l l}
00 & OKAY \\ 
10 & SLVERR \\
11 & DECERR \\
\end{tabular}
\end{fielddesc}

A \code{DECERR} indicates an unknown address, and a \code{SLVERR} a command
error, eg writing to a read-only register. The value in a reset response is the
FPGA features register.

\subsubsection{Addresses}

\begin{regfields}
\bitfield{\makebox[0.125\textwidth][c]{route}}{31}{24}
\bitfield{\makebox[0.5\textwidth][c]{24 bit address}}{23}{0}
\end{regfields}

The MSB of the address is used to route the command to various sub-systems
within the design

\begin{fielddesc}
\begin{tabular}{l l}
"0C" & registers for channel "C" \\ 
"10" & global registers \\
"20" & SPI communication \\
\end{tabular}
\end{fielddesc}

the SPI channels communicate with the chips on FMC daughter board,
the four ADS62P49 ADC chips and the Analogue Devices AD9510 clock distribution
chip. SPI address are of the form

\begin{regfields}
\bitfield{\makebox[30pt]{20}}{31}{24}
\bitfield{reserved}{23}{16}
\bitfield{chip select}{15}{8}
\bitfield{address}{7}{0}
\end{regfields}

with a FMC108 installed the chip selects are

\begin{fielddesc}
\begin{tabular}{l l}
bit 0 & ADS62P49 chip 0 (channels 0 and 1)\\
bit 1 & ADS62P49 chip 1 (channels 2 and 3)\\
bit 2 & ADS62P49 chip 2 (channels 4 and 5)\\
bit 3 & ADS62P49 chip 3 (channels 6 and 7)\\
bit 4 & AD9510 chip \\
\end{tabular}
\end{fielddesc}

\clearpage
\subsection{Global registers}

\begin{register}{cpu\_version}{01000000 READ ONLY}
\bitfield{ year }{31}{28}
\bitfield{month}{27}{24}
\bitfield{ day }{23}{16}
\bitfield{hour}{15}{8}
\bitfield{minute}{7}{0}
\end{register}
The build date for the software running on the main picoblaze CPU that controls
register IO, \code{year} is modulo 16.

\begin{register}{HDL version}{01000001}
	\bitfield{reserved}{31}{24}
	\bitfield{git short SHA-1}{23}{0}
\end{register}
The short SHA-1 for the git commit of the HDL source.

\begin{register}{FPGA features}{01800000 READ ONLY}
\bitflag{AD9510 status}{10}
\bitflag{FMC power}{9}
\bitflag{ FMC }{8}
\bitfield{ ADC }{7}{4}
\bitfield{Channels}{3}{0}
\end{register}

\begin{fielddesc}
\begin{tabular}{l l}
AD9510 status & See data sheet \\
FMC power & FMC card booted with out error and is powered up\\
FMC present & FMC card is installed \\ 
ADC & number of adc chips on the FMC card\\
DSP & number of processing channels instantiated in the FPGA.\\
\end{tabular}
\end{fielddesc}

\begin{register}{ADC enable}{01000080}
\bitfield{reserved}{31}{8}
\bitfield{enable bits}{7}{0}
\end{register}
Each enable bit corresponds to the same ADC channel number, disabling an ADC
channel puts it in stand-by mode.

\begin{register}{Channel enable}{01000100}
\bitfield{reserved}{31}{8}
\bitfield{enable bits}{7}{0}
\end{register}
Each enable bit corresponds to the same processing channel, disabling a channel
prevents any information from that channel appearing in the event stream.

\begin{register}{Tick period}{01000010}
\bitfield{tick period}{31}{0}
\end{register}
The length of time between tick events in system clocks ($4 ns$).

\begin{register}{Tick latency}{01000020}
\bitfield{tick latency}{31}{0}
\end{register}
The maximum allowed to elapse since the last tick event was put in the event
stream before the event buffer is flushed.

\section{VHDL Notes}

\begin{verbatim}
Reset sequence
CPU reset - main CPU
reset0    - AD9510, ADCs, ADC FIFOs, 
            Global and channel registers, 
            Channel CPUs, ethernet_framer
reset1    - eventstream_mux, MCA
reset2    - measurement_unit
\end{verbatim}

\end{document}
