\documentclass{beamer}

\usepackage{svg}
\setsvg{inkscape = inkscape -z -D}
\setsvg{svgpath = ../docs/images/}
\usepackage{relsize}
\usepackage{subfloat}
\usepackage{graphicx}
\usepackage{amsmath}
\usepackage{bitpattern}
\usepackage{bytefield}
% This file is a solution template for:

% - Talk at a conference/colloquium.
% - Talk length is about 20min.
% - Style is ornate.

% Copyright 2004 by Till Tantau <tantau@users.sourceforge.net>.
%
% In principle, this file can be redistributed and/or modified under
% the terms of the GNU Public License, version 2.
%
% However, this file is supposed to be a template to be modified
% for your own needs. For this reason, if you use this file as a
% template and not specifically distribute it as part of a another
% package/program, I grant the extra permission to freely copy and
% modify this file as you see fit and even to delete this copyright
% notice. 


\mode<presentation>
{
  \usetheme{Warsaw}
  % or ...

  \setbeamercovered{transparent}
  % or whatever (possibly just delete it)
}


\usepackage[english]{babel}
% or whatever

\usepackage[latin1]{inputenc}
% or whatever

\usepackage{times}
\usepackage[T1]{fontenc}
% Or whatever. Note that the encoding and the font should match. If T1
% does not look nice, try deleting the line with the fontenc.


\title[TES Processor] % (optional, use only with long paper titles)
{Real time processing of Transition edge sensor (TES) data}

%\subtitle
%{Include Only If Paper Has a Subtitle}

\author[Geoff Gillett] % (optional, use only with lots of authors)
{Geoff Gillett}
% - Give the names in the same order as the appear in the paper.
% - Use the \inst{?} command only if the authors have different
%   affiliation.

\institute[The University of Queensland] % (optional, but mostly needed)
{
  ARC Centre of excellence for Engineered Quantum Systems\\
  ARC Centre of excellence for Quantum Computing and Communication Technology\\
  The University of Queensland}
% - Use the \inst command only if there are several affiliations.
% - Keep it simple, no one is interested in your street address.

\date[Thesis review 2016] % (optional, should be abbreviation of conference
% name)
{2016}
% - Either use conference name or its abbreviation.
% - Not really informative to the audience, more for people (including
%   yourself) who are reading the slides online

\subject{Physics/Engineering}
% This is only inserted into the PDF information catalog. Can be left
% out. 


% If you have a file called "university-logo-filename.xxx", where xxx
% is a graphic format that can be processed by latex or pdflatex,
% resp., then you can add a logo as follows:

% \pgfdeclareimage[height=0.5cm]{university-logo}{university-logo-filename}
% \logo{\pgfuseimage{university-logo}}

% Delete this, if you do not want the table of contents to pop up at
% the beginning of each subsection:
% \AtBeginSubsection[]
% {
%   \begin{frame}<beamer>{Outline}
%     \tableofcontents[currentsection,currentsubsection]
%   \end{frame}
% }


% If you wish to uncover everything in a step-wise fashion, uncomment
% the following command: 

%\beamerdefaultoverlayspecification{<+->}


\begin{document}

\begin{frame}
  \titlepage
\end{frame}

\begin{frame}{Outline}
  \tableofcontents
  % You might wish to add the option [pausesections]
\end{frame}


% Structuring a talk is a difficult task and the following structure
% may not be suitable. Here are some rules that apply for this
% solution: 

% - Exactly two or three sections (other than the summary).
% - At *most* three subsections per section.
% - Talk about 30s to 2min per frame. So there should be between about
%   15 and 30 frames, all told.

% - A conference audience is likely to know very little of what you
%   are going to talk about. So *simplify*!
% - In a 20min talk, getting the main ideas across is hard
%   enough. Leave out details, even if it means being less precise than
%   you think necessary.
% - If you omit details that are vital to the proof/implementation,
%   just say so once. Everybody will be happy with that.


\section{TES Physics}
\begin{frame}{Calorimeters}
  \begin{figure}[!hpbt]
    \centering
    \includesvg[width=0.8\textwidth]{calorimeter}
  \end{figure}
  \vspace{-1 cm}
  {\Tiny Lita \emph{et.al} Counting near-infrared
  single-photons with 95\% efficiency. doi:10.1364/OE.16.003032}
\end{frame}

\begin{frame}{Type II superconductor}
  \begin{figure}[!hpbt]
    \centering
    \includesvg[width=0.8\textwidth]{typeIIsuperconductor}
  \end{figure}
\end{frame}

\begin{frame}{Biasing and Readout}
  \begin{figure}[!hpbt]
    \subfloat[Bias and readout cicuit]{%
       \includesvg[pretex=\relscale{0.5},width=0.5\textwidth]{circuit}
       \label{fig:circuit}
    }
    \subfloat[TES biasing]{
      \includesvg[width=0.5\textwidth]{sensorbias}
      \label{fig:sensorbias}
    }
  \end{figure}
\end{frame}

\begin{frame}{Photon detection}
  \begin{figure}[!hpbt]
    \centering 
    \includesvg[width=0.7\textwidth]{detection}
  \end{figure}
  \vspace{-0.8 cm}
  {\Tiny Levine \emph{et.el.} Photon-number uncertainty in a superconducting 
  transition edge sensor sensor beyond resolved-photon-number determination.  
  doi:10.1364/JOSAB.31.000B20}
\end{frame}
  
\section{Field Programmable Gate Array}
\begin{frame}{Field Programmable Gate Array (FPGA)}
  \begin{itemize}
    \item Reconfigurable Digital Circuit
    \item Array of resources RAM, DSP, LUT, Flip-flops etc
    \item Circuit described using a Hardware Description language (HDL) 
  \end{itemize}
\end{frame}

\section{Processor design}

\subsection{Signal Conditioning}

\begin{frame}{Signal Conditioning}
  \begin{figure}[!hpbt]
    \centering
    \includegraphics[width=0.9\textwidth]{../docs/images/pipeline.png}
  \end{figure}
\end{frame}

\begin{frame}{Baseline Correction}
  \begin{figure}[!hpbt]
    \centering
    \includegraphics[width=0.8\textwidth]{../docs/images/baselineestimation.png}
  \end{figure}
\end{frame}

\begin{frame}{Digital Filters}
  Linear time-invariant (LTI) systems - output is convolution of input and impulse
  response \\
  \vspace{0.5cm}
  Discrete convolution: $y[n]=\sum_{k=0}^{N-1}h_k x[n-k]$ \\
  \begin{figure}[!hpbt]
    \centering 
    \includesvg[width=0.8\textwidth]{FIR}
  \end{figure}
  \begin{center}
    Direct form Finite Impulse Response (FIR) filter
  \end{center}
\end{frame}

\begin{frame}{Default Magnitude Responses}
  \begin{figure}[!hpbt]
    \subfloat[Smoother]{%
       \includesvg[pretex=\relscale{0.5},width=0.5\textwidth]{lowpass_resp}
    }
    \subfloat[Differentiator]{
      \includesvg[width=0.5\textwidth]{diff_resp}
    }
  \end{figure}
\end{frame}

\subsection{Measurement}

\bpLittleEndian
\bpNumberBitsAbove
\bpSetTickHeight{1pt}
\bpSetBitWidth{8pt}
\renewcommand\bpFormatBitNumber[1]{{\Tiny\ttfamily\strut#1}}
\renewcommand\bpFormatField[1]{{\tiny\ttfamily\strut#1}}

\newcommand{\colorbitbox}[3]{%
  \rlap{\bitbox{#2}{\color{#1}\rule{\width}{\height}}}%
  \bitbox{#2}{#3}}

\definecolor{lightred}{rgb}{1,0.7,0.71}

\begin{frame}{Event stream}

  {\scriptsize Peak event:} \\
  \bitpattern[startBit=63]{height}[16][8]{rise time}[16][8]{eFlags}[16][8]
  {rTime}[16][8]/
  \\
  {\scriptsize Area event:} \\
  \bitpattern[startBit=63]{area}[32][16]{eFlags}[16][8]{rTime}[16][8]/
  \\
  {\scriptsize Pulse event:} \\
  \bitpattern[startBit=63]{size}[16][8]{plength}[16][8]{eFlags}[16][8]
  {rTime}[16][8]/
  \\
  \bitpattern[startBit=63,noBitNumbers]{area}[32][16]{pthresh}[16][8]
  {sThresh}[16][8]/
  \\
  \bitpattern[startBit=63,noBitNumbers]{height}[16][8]{minima}[16][8]
  {rise time}[16][8]{pTime}[16][8]/
  {\scriptsize peak}
  \\
  \makebox[20 pt]{} {\tiny Number of peak slots set by \code{max\_peaks} register}
  \bitpattern[startBit=63,noBitNumbers]{height}[16][8]{minima}[16][8]
  {rise time}[16][8]{pTime}[16][8]/
  \\
  {\scriptsize Tick event:}\\
  \bitpattern[startBit=63]{period}[32][16]{tFlags}[16][8]{rTime}[16][8]/
  \bitpattern[startBit=63,noBitNumbers]{full time-stamp}[64][32]/
  \bitpattern[startBit=63,noBitNumbers]{overflow and error flags}[64][32]/
  \\
\end{frame}

\begin{frame}{Peak Extraction:Relative measurement flag-0}
  \tiny{}
  \begin{bytefield}[bitheight={10pt},
                    boxformatting={\centering\tiny},
                    bitformatting={\TINY}]{16}
     \bitheader[endianness=big]{0-15} \\
     %\vspace{-1mm}
     \bitbox{4}{PN} 
     \colorbitbox{lightred}{1}{R} 
     \bitbox{3}{C} 
     \bitbox{2}{TT} 
     \bitbox{2}{HT} 
     \bitbox{2}{DT} 
     \bitbox{1}{T} 
     \bitbox{1}{N}
  \end{bytefield}
  \raisebox{3 pt}{eFlags}
  
  \begin{bytefield}[bitheight={10pt},
                    boxformatting={\centering\tiny},
                    bitformatting={\TINY}]{16}
     \\
     %\vspace{-1mm}
     \bitbox{4}{0} 
     \colorbitbox{lightred}{1}{0} 
     \bitbox{3}{0} 
     \bitbox{2}{3} 
     \bitbox{2}{0} 
     \bitbox{2}{0} 
     \bitbox{1}{0} 
     \bitbox{1}{?} 
  \end{bytefield}
  
  \vspace{-0.5cm}
  
  \begin{figure}[!hpbt]
    \centering
    \includesvg[width=0.7\textwidth]{peakextraction1}
  \end{figure}
  
  \vspace{-0.5cm}
  
  \begin{center}
    \bitpattern[startBit=63]{1761}[16][8]{134}[16][8]{eFlags}[16][8]
    {0}[16][8]/
  \end{center}
\end{frame}

\begin{frame}{Peak Extraction:Relative measurement flag-1}
  \tiny{}
  \begin{bytefield}[bitheight={10pt},
                    boxformatting={\centering\tiny},
                    bitformatting={\TINY}]{16}
     \bitheader[endianness=big]{0-15} \\
     %\vspace{-1mm}
     \bitbox{4}{PN} 
     \colorbitbox{lightred}{1}{R} 
     \bitbox{3}{C} 
     \bitbox{2}{TT} 
     \bitbox{2}{HT} 
     \bitbox{2}{DT} 
     \bitbox{1}{T} 
     \bitbox{1}{N}
  \end{bytefield}
  \raisebox{3 pt}{eFlags}
  
  \begin{bytefield}[bitheight={10pt},
                    boxformatting={\centering\tiny},
                    bitformatting={\TINY}]{16}
     \\
     %\vspace{-1mm}
     \bitbox{4}{0} 
     \colorbitbox{lightred}{1}{1} 
     \bitbox{3}{0} 
     \bitbox{2}{3} 
     \bitbox{2}{0} 
     \bitbox{2}{0} 
     \bitbox{1}{0} 
     \bitbox{1}{?} 
  \end{bytefield}
  
  \vspace{-0.5cm}
  
  \begin{figure}[!hpbt]
    \centering
    \includesvg[width=0.7\textwidth]{peakextraction2}
  \end{figure}
  
  \vspace{-0.5cm}
  
  \begin{center}
    \bitpattern[startBit=63]{1814}[16][8]{134}[16][8]{eFlags}[16][8]
    {0}[16][8]/
  \end{center}
\end{frame}


\begin{frame}{Peak Extraction: Timing type-CF low}
  \tiny{}
  \begin{bytefield}[bitheight={10pt},
                    boxformatting={\centering\tiny},
                    bitformatting={\TINY}]{16}
     \bitheader[endianness=big]{0-15} \\
     %\vspace{-1mm}
     \bitbox{4}{PN} 
     \bitbox{1}{R} 
     \bitbox{3}{C} 
     \colorbitbox{lightred}{2}{TT} 
     \bitbox{2}{HT} 
     \bitbox{2}{DT} 
     \bitbox{1}{T} 
     \bitbox{1}{N}
  \end{bytefield}
  \raisebox{3 pt}{eFlags}
  
  \begin{bytefield}[bitheight={10pt},
                    boxformatting={\centering\tiny},
                    bitformatting={\TINY}]{16}
     \\
     %\vspace{-1mm}
     \bitbox{4}{0} 
     \bitbox{1}{1} 
     \bitbox{3}{0} 
     \colorbitbox{lightred}{2}{2} 
     \bitbox{2}{0} 
     \bitbox{2}{0} 
     \bitbox{1}{0} 
     \bitbox{1}{?} 
  \end{bytefield}
  
  \vspace{-0.5cm}
  
  \begin{figure}[!hpbt]
    \centering
    \includesvg[width=0.7\textwidth]{peakextraction3}
  \end{figure}
  
  \vspace{-0.5cm}
  
  \begin{center}
    \bitpattern[startBit=63]{1622}[16][8]{97}[16][8]{eFlags}[16][8]
    {+37}[16][8]/
  \end{center}
\end{frame}

\begin{frame}{Peak Extraction: Timing type-slope crossing}
  \tiny{}
  \begin{bytefield}[bitheight={10pt},
                    boxformatting={\centering\tiny},
                    bitformatting={\TINY}]{16}
     \bitheader[endianness=big]{0-15} \\
     %\vspace{-1mm}
     \bitbox{4}{PN} 
     \bitbox{1}{R} 
     \bitbox{3}{C} 
     \colorbitbox{lightred}{2}{TT} 
     \bitbox{2}{HT} 
     \bitbox{2}{DT} 
     \bitbox{1}{T} 
     \bitbox{1}{N}
  \end{bytefield}
  \raisebox{3 pt}{eFlags}
  
  \begin{bytefield}[bitheight={10pt},
                    boxformatting={\centering\tiny},
                    bitformatting={\TINY}]{16}
     \\
     %\vspace{-1mm}
     \bitbox{4}{0} 
     \bitbox{1}{1} 
     \bitbox{3}{0} 
     \colorbitbox{lightred}{2}{1} 
     \bitbox{2}{0} 
     \bitbox{2}{0} 
     \bitbox{1}{0} 
     \bitbox{1}{?} 
  \end{bytefield}
  
  \vspace{-0.5cm}
  
  \begin{figure}[!hpbt]
    \centering
    \includesvg[width=0.7\textwidth]{peakextraction4}
  \end{figure}
  
  \vspace{-0.5cm}
  
  \begin{center}
    \bitpattern[startBit=63]{1502}[16][8]{90}[16][8]{eFlags}[16][8]
    {+44}[16][8]/
  \end{center}
\end{frame}

\begin{frame}{Peak Extraction: Timing type-pulse crossing}
  \tiny{}
  \begin{bytefield}[bitheight={10pt},
                    boxformatting={\centering\tiny},
                    bitformatting={\TINY}]{16}
     \bitheader[endianness=big]{0-15} \\
     %\vspace{-1mm}
     \bitbox{4}{PN} 
     \bitbox{1}{R} 
     \bitbox{3}{C} 
     \colorbitbox{lightred}{2}{TT} 
     \bitbox{2}{HT} 
     \bitbox{2}{DT} 
     \bitbox{1}{T} 
     \bitbox{1}{N}
  \end{bytefield}
  \raisebox{3 pt}{eFlags}
  
  \begin{bytefield}[bitheight={10pt},
                    boxformatting={\centering\tiny},
                    bitformatting={\TINY}]{16}
     \\
     %\vspace{-1mm}
     \bitbox{4}{0} 
     \bitbox{1}{1} 
     \bitbox{3}{0} 
     \colorbitbox{lightred}{2}{0} 
     \bitbox{2}{0} 
     \bitbox{2}{0} 
     \bitbox{1}{0} 
     \bitbox{1}{?} 
  \end{bytefield}
  
  \vspace{-0.5cm}
  
  \begin{figure}[!hpbt]
    \centering
    \includesvg[width=0.7\textwidth]{peakextraction5}
  \end{figure}
  
  \vspace{-0.5cm}
  
  \begin{center}
    \bitpattern[startBit=63]{1161}[16][8]{58}[16][8]{eFlags}[16][8]
    {+76}[16][8]/
  \end{center}
\end{frame}

\begin{frame}{Peak Extraction: Height type-CF high}
  \tiny{}
  \begin{bytefield}[bitheight={10pt},
                    boxformatting={\centering\tiny},
                    bitformatting={\TINY}]{16}
     \bitheader[endianness=big]{0-15} \\
     %\vspace{-1mm}
     \bitbox{4}{PN} 
     \bitbox{1}{R} 
     \bitbox{3}{C} 
     \bitbox{2}{TT} 
     \colorbitbox{lightred}{2}{HT} 
     \bitbox{2}{DT} 
     \bitbox{1}{T} 
     \bitbox{1}{N}
  \end{bytefield}
  \raisebox{3 pt}{eFlags}
  
  \begin{bytefield}[bitheight={10pt},
                    boxformatting={\centering\tiny},
                    bitformatting={\TINY}]{16}
     \\
     %\vspace{-1mm}
     \bitbox{4}{0} 
     \bitbox{1}{1} 
     \bitbox{3}{0} 
     \bitbox{2}{2} 
     \colorbitbox{lightred}{2}{1} 
     \bitbox{2}{0} 
     \bitbox{1}{0} 
     \bitbox{1}{?} 
  \end{bytefield}
  
  \vspace{-0.5cm}
  
  \begin{figure}[!hpbt]
    \centering
    \includesvg[width=0.7\textwidth]{peakextraction6}
  \end{figure}
  
  \vspace{-0.5cm}
  
  \begin{center}
    \bitpattern[startBit=63]{1433}[16][8]{69}[16][8]{eFlags}[16][8]
    {+37}[16][8]/
  \end{center}
\end{frame}

\begin{frame}{Peak Extraction: Height type-slope integral}
  \tiny{}
  \begin{bytefield}[bitheight={10pt},
                    boxformatting={\centering\tiny},
                    bitformatting={\TINY}]{16}
     \bitheader[endianness=big]{0-15} \\
     %\vspace{-1mm}
     \bitbox{4}{PN} 
     \bitbox{1}{R} 
     \bitbox{3}{C} 
     \bitbox{2}{TT} 
     \colorbitbox{lightred}{2}{HT} 
     \bitbox{2}{DT} 
     \bitbox{1}{T} 
     \bitbox{1}{N}
  \end{bytefield}
  \raisebox{3 pt}{eFlags}
  
  \begin{bytefield}[bitheight={10pt},
                    boxformatting={\centering\tiny},
                    bitformatting={\TINY}]{16}
     \\
     %\vspace{-1mm}
     \bitbox{4}{0} 
     \bitbox{1}{1} 
     \bitbox{3}{0} 
     \bitbox{2}{2} 
     \colorbitbox{lightred}{2}{2} 
     \bitbox{2}{0} 
     \bitbox{1}{0} 
     \bitbox{1}{?} 
  \end{bytefield}
  
  \vspace{-0.5cm}
  
  \begin{figure}[!hpbt]
    \centering
    \includesvg[width=0.7\textwidth]{peakextraction7}
  \end{figure}
  
  \vspace{-0.5cm}
  
  \begin{center}
    \bitpattern[startBit=63]{226.0}[16][8]{97}[16][8]{eFlags}[16][8]
    {+37}[16][8]/
  \end{center}
\end{frame}

\bpSetTickHeight{1pt}
\bpSetBitWidth{8pt}
\renewcommand\bpFormatBitNumber[1]{{\Tiny\ttfamily\strut#1}}
\renewcommand\bpFormatField[1]{{\Tiny\ttfamily\strut#1}}
\begin{frame}{Pulse Measurement}
  \tiny{}
  \begin{bytefield}[bitheight={10pt},
                    boxformatting={\centering\tiny},
                    bitformatting={\TINY}]{16}
     \bitheader[endianness=big]{0-15} \\
     %\vspace{-1mm}
     \bitbox{4}{PN} 
     \bitbox{1}{R} 
     \bitbox{3}{C} 
     \bitbox{2}{TT} 
     \bitbox{2}{HT} 
     \colorbitbox{lightred}{2}{DT} 
     \bitbox{1}{T} 
     \bitbox{1}{N}
  \end{bytefield}
  \raisebox{3 pt}{eFlags}
  
  \begin{bytefield}[bitheight={10pt},
                    boxformatting={\centering\tiny},
                    bitformatting={\TINY}]{16}
     \\
     %\vspace{-1mm}
     \bitbox{4}{1} 
     \bitbox{1}{1} 
     \bitbox{3}{0} 
     \bitbox{2}{2} 
     \bitbox{2}{0} 
     \colorbitbox{lightred}{2}{2} 
     \bitbox{1}{0} 
     \bitbox{1}{?} 
  \end{bytefield}
  
  \vspace{-0.5cm}
  \begin{figure}[!hpbt]
    \centering
    \includesvg[width=0.6\textwidth]{pulsemeasurement}
  \end{figure}
  \vspace{-0.5cm}
  %{\Tiny Pulse event:} \\
  \bitpattern[startBit=63]{size}[16][8]{plength}[16][8]{eFlags}[16][8]
  {rTime}[16][8]/
  \\
  \bitpattern[startBit=63,noBitNumbers]{area}[32][16]{pthresh}[16][8]
  {sThresh}[16][8]/
  \\
  \bitpattern[startBit=63,noBitNumbers]{height}[16][8]{minima}[16][8]
  {rise time}[16][8]{pTime}[16][8]/
  \\
  \bitpattern[startBit=63,noBitNumbers]{height}[16][8]{minima}[16][8]
  {rise time}[16][8]{pTime}[16][8]/
\end{frame}

\subsection{Infrastructure}

\begin{frame}{Infrastructure}
  \begin{figure}[!hpbt]
    \centering 
    \includesvg[pretex=\relscale{0.3},width=0.5\textwidth]{design}
  \end{figure}
\end{frame}

\subsection{Status}

\begin{frame}{Project status}
  Very much still a work in progress
  \begin{itemize}
    \item Pre-amp board
    \item field testing hardware
    \item Host computer software
  \end{itemize}
  Thank You
\end{frame}
\end{document}



\subsection{Previous Work}

\begin{frame}{Make Titles Informative.}
\end{frame}

\begin{frame}{Make Titles Informative.}
\end{frame}



\section{Our Results/Contribution}

\subsection{Main Results}

\begin{frame}{Make Titles Informative.}
\end{frame}

\begin{frame}{Make Titles Informative.}
\end{frame}

\begin{frame}{Make Titles Informative.}
\end{frame}


\subsection{Basic Ideas for Proofs/Implementation}

\begin{frame}{Make Titles Informative.}
\end{frame}

\begin{frame}{Make Titles Informative.}
\end{frame}

\begin{frame}{Make Titles Informative.}
\end{frame}



\section*{Summary}

\begin{frame}{Summary}

  % Keep the summary *very short*.
  \begin{itemize}
  \item
    The \alert{first main message} of your talk in one or two lines.
  \item
    The \alert{second main message} of your talk in one or two lines.
  \item
    Perhaps a \alert{third message}, but not more than that.
  \end{itemize}
  
  % The following outlook is optional.
  \vskip0pt plus.5fill
  \begin{itemize}
  \item
    Outlook
    \begin{itemize}
    \item
      Something you haven't solved.
    \item
      Something else you haven't solved.
    \end{itemize}
  \end{itemize}
\end{frame}



% All of the following is optional and typically not needed. 
\appendix
\section<presentation>*{\appendixname}
\subsection<presentation>*{For Further Reading}

\begin{frame}[allowframebreaks]
  \frametitle<presentation>{For Further Reading}
    
  \begin{thebibliography}{10}
    
  \beamertemplatebookbibitems
  % Start with overview books.

  \bibitem{Author1990}
    A.~Author.
    \newblock {\em Handbook of Everything}.
    \newblock Some Press, 1990.
 
    
  \beamertemplatearticlebibitems
  % Followed by interesting articles. Keep the list short. 

  \bibitem{Someone2000}
    S.~Someone.
    \newblock On this and that.
    \newblock {\em Journal of This and That}, 2(1):50--100,
    2000.
  \end{thebibliography}
\end{frame}



